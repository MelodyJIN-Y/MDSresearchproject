\chapter{Assumptions}
\section{Assumption}
The experiments are conducted with simulated count data from Splatter which makes assumptions about the expected behaviour of true DEs. 
Splatter fits the real count matrix with a lognormal distribution and allows . The parameter de.facloc ad de.facscale control the mean and standard deviation of the assigned multipliers to the ramdomly selected true DEs. 
Although two-sided tests are popular in practice, one sided test will be used to identify up-regulated genes. The purpose of marker genes analysis is to identify up-regulated genes that can be represntative for specific cell types so that more clues can be made about existing cell types. Two-sided statistical test will include both up-regulated and down-regulated genes, and those down-regualted genes will cause ambigurity for further interpretation. 
\section{Limma Framework}
Limma fits a linear model for each gene. The adjusted p-values are believed to be over-optimistic. 
\section{Permutation Framework}
For any statistic $\textbf{T}_{obs}$ calculated from original count matrix, the same statistic $\textbf{T}_{perm}$ will be constructed from each permutated count matrix. The one-sided p-value will be evaluated based on the probability of having a larger permutated statitic than the observed statistic.\\
Let $N$ denote the number of permutations, the raw p-value $p$ for each gene will be 
\begin{align*} 
	p &=  \frac{\textbf{I}_{\textbf{T}_{perm} > \textbf{T}_{obs}} + 1}{N+1} 
\end{align*}
The key advantage of permutation test is the distribution-free property, which makes a wide range of statistics be and comparable.   
As the efficiency of limma framework, the permutation framework is implementated on top of the limma package, which makes the comparision simpler. 
%\subsection{Dataset }
\section{statistic}
t-statistic ($t$) and moderated t-statistic \modt are widely used for identifying marker genes in scRNA-seq. 
It is agued that \modt should be prefera
treat.t 
One additional statistic, $\textbf{W}, ~ \textbf{W} = \log \mathrm{FC} \times (1 - p) $ where $p$ is the raw p-value of the t-statistic. 
An ideal marker gene is expected to have a larger logFC, and a smaller p-value, which will generate a larger $W$ statistic. 