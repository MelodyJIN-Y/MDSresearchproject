%%
%% Modified by Ricardo Garcia-Rosas to satisfy the rules established by the University of Melbourne's Research Higher Degrees Committee as of 4th of June 2019.
%% Guidelines can be found at: https://gradresearch.unimelb.edu.au/__data/assets/pdf_file/0004/2027839/Preparation-of-GR-theses-rules.pdf
%%
%%%%%%%%%%%%%%%%%%%%%%%%%%%%%%%%%%%%%%%%%%%%%%%%%%%%%%%%%%%%%%%%%%%%%%%%%
%% IMPORTANT NOTE TO AUTHOR:
%% As part of the guidelines, the use of the university logo is not permitted. This template contains it to make it easier to find/recognise in the Overleaf Gallery. To make the template compliant please go to 'Thesis.cls' and comment out the \includegraphics command in line 217 (it is clearly highlited).
%%%%%%%%%%%%%%%%%%%%%%%%%%%%%%%%%%%%%%%%%%%%%%%%%%%%%%%%%%%%%%%%%%%%%%%%%
%%
%% ----------------------------------------------------------------
%% Thesis.tex -- MAIN FILE (the one that you compile with LaTeX)
%% ---------------------------------------------------------------- 

% Set up the document
\documentclass[a4paper, 11pt, oneside]{Thesis}  % Use the "Thesis" style, based on the ECS Thesis style by Steve Gunn
%
% Put your figures in this directory
\graphicspath{Figures/}  % Location of the graphics files (set up for graphics to be in PDF format)
%

% Include any extra LaTeX packages required
\usepackage[square, numbers, comma, sort&compress]{natbib}  % Use the "Natbib" style for the references in the Bibliography
\usepackage{verbatim}  % Needed for the "comment" environment to make LaTeX comments
\usepackage{vector}  % Allows "\bvec{}" and "\buvec{}" for "blackboard" style bold vectors in maths
\hypersetup{urlcolor=blue, colorlinks=true}  % Colours hyperlinks in blue, but this can be distracting if there are many links.

%% ----------------------------------------------------------------
\begin{document}
\frontmatter      % Begin Roman style (i, ii, iii, iv...) page numbering

%
\UNIVERSITY{{THE UNIVERSITY OF MELBOURNE }}    
%
%%%%%%%%%%%%%%%%%%%%%%%%%%%%%%%%%%%%%%%%%%%%%%%%%%%%%%%%%%%%%%%%%%%%%%%%%
% Update your department and school here:
\department{{Department of Mad Science}}
\school{{Melbourne School of Awesome}}
%%%%%%%%%%%%%%%%%%%%%%%%%%%%%%%%%%%%%%%%%%%%%%%%%%%%%%%%%%%%%%%%%%%%%%%%%

%
%%%%%%%%%%%%%%%%%%%%%%%%%%%%%%%%%%%%%%%%%%%%%%%%%%%%%%%%%%%%%%%%%%%%%%%%%
% Set up the Title Page
% Change your thesis title and your information here
\title  {Thesis Title}
\authors  {\texorpdfstring
            {\href{your web site or email address}{Author Name\\ \small ORCID: }}
            {Author Name}
            }
\addresses  {\groupname\\\deptname\\\univname}  % Do not change this here, instead these must be set in the "Thesis.cls" file, please look through it instead
\date       {\today}
\subject    {}
\keywords   {}
%%%%%%%%%%%%%%%%%%%%%%%%%%%%%%%%%%%%%%%%%%%%%%%%%%%%%%%%%%%%%%%%%%%%%%%%%

\maketitle
%% ----------------------------------------------------------------

\setstretch{1.3}  % It is better to have smaller font and larger line spacing than the other way round

% Define the page headers using the FancyHdr package and set up for one-sided printing
\fancyhead{}  % Clears all page headers and footers
\rhead{\thepage}  % Sets the right side header to show the page number
\lhead{}  % Clears the left side page header

\pagestyle{fancy}  % Finally, use the "fancy" page style to implement the FancyHdr headers

%% ----------------------------------------------------------------
% The Abstract Page

\addtotoc{Abstract}  % Add the "Abstract" page entry to the Contents
\abstract{
\addtocontents{toc}{\vspace{1em}}  % Add a gap in the Contents, for aesthetics

The Thesis Abstract is written here (and usually kept to just this page). The page is kept centered vertically so can expand into the blank space above the title too\ldots

}
\clearpage  % Abstract ended, start a new page

%% ----------------------------------------------------------------
% Declaration Page required for the Thesis, your institution may give you a different text to place here
%
% Guidelines as of 2019/06/04
% https://gradresearch.unimelb.edu.au/__data/assets/pdf_file/0004/2027839/Preparation-of-GR-theses-rules.pdf
%
\Declaration{

\addtocontents{toc}{\vspace{1em}}  % Add a gap in the Contents, for aesthetics

I, Xinyi Jin, declare that this thesis titled, `THESIS TITLE' and the work presented in it are my own. I confirm that:

\begin{itemize} 
\item[\tiny{$\blacksquare$}] The thesis comprises only my original work towards the NAME OF AWARD except where indicated in the preface;
 
\item[\tiny{$\blacksquare$}] due acknowledgement has been made in the text to all other material used; and

\item[\tiny{$\blacksquare$}] the thesis is fewer than the maximum word limit in length, exclusive of tables, maps, bibliographies and appendices as approved by the Research Higher Degrees Committee.
\\
\end{itemize}
 
 
Signed:\\
\rule[1em]{25em}{0.5pt}  % This prints a line for the signature
 
Date:\\
\rule[1em]{25em}{0.5pt}  % This prints a line to write the date
}
\clearpage  % Declaration ended, now start a new page

%% ----------------------------------------------------------------
% Preface Page required for the Thesis, your institution may give you a different text to place here
%
% Guidelines as of 2019/06/04
% https://gradresearch.unimelb.edu.au/__data/assets/pdf_file/0004/2027839/Preparation-of-GR-theses-rules.pdf
%
\Preface{

\addtocontents{toc}{\vspace{1em}}  % Add a gap in the Contents, for aesthetics

Where applicable, the following information must be included in a preface:
\begin{itemize}
\item[\tiny{$\blacksquare$}] a description of work towards the thesis that was carried out in collaboration with others, indicating the nature and proportion of the contribution of others and in general terms the portions of the work which the student claims as original;
\item[\tiny{$\blacksquare$}] a description of work towards the thesis that has been submitted for other qualifications;
\item[\tiny{$\blacksquare$}] a description of work towards the thesis that was carried out prior to enrolment in the degree;
\item[\tiny{$\blacksquare$}] whether any third party editorial assistance was provided in preparation of
the thesis and whether the persons providing this assistance are knowledgeable in the academic discipline of the thesis;
\item[\tiny{$\blacksquare$}] the contributions of all persons involved in any multi-authored publications or
articles in preparation included in the thesis;
\item[\tiny{$\blacksquare$}] the publication status of all chapters presented in article format using the
descriptors below;
    \begin{itemize}
        \item Unpublished material not submitted for publication
        \item Submitted for publication to [publication name] on [date]
        \item In revision following peer review by [publication name]
        \item Accepted for publication by [publication name] on [date]
        \item Published by [publication name] on [date]
    \end{itemize}
\item[\tiny{$\blacksquare$}] an acknowledgement of all sources of funding, including grant identification
numbers where applicable and Australian Government Research Training Program Scholarships, including fee offset scholarships.
\end{itemize}

}
\clearpage  % Preface ended, now start a new page

%% ----------------------------------------------------------------
% The Acknowledgements page, for thanking everyone
\setstretch{1.3}  % Reset the line-spacing to 1.3 for body text (if it has changed)
\acknowledgements{
\addtocontents{toc}{\vspace{1em}}  % Add a gap in the Contents, for aesthetics

The acknowledgements and the people to thank go here, don't forget to include your project advisor\ldots

}
\clearpage  % End of the Acknowledgements
%% ----------------------------------------------------------------

\pagestyle{fancy}  %The page style headers have been "empty" all this time, now use the "fancy" headers as defined before to bring them back


%% ----------------------------------------------------------------
\lhead{\emph{Contents}}  % Set the left side page header to "Contents"
\tableofcontents  % Write out the Table of Contents

%% ----------------------------------------------------------------
\lhead{\emph{List of Figures}}  % Set the left side page header to "List if Figures"
\listoffigures  % Write out the List of Figures

%% ----------------------------------------------------------------
\lhead{\emph{List of Tables}}  % Set the left side page header to "List of Tables"
\listoftables  % Write out the List of Tables

%% ----------------------------------------------------------------
\setstretch{1.5}  % Set the line spacing to 1.5, this makes the following tables easier to read
\clearpage  % Start a new page
\lhead{\emph{Abbreviations}}  % Set the left side page header to "Abbreviations"
\listofsymbols{ll}  % Include a list of Abbreviations (a table of two columns)
{
% \textbf{Acronym} & \textbf{W}hat (it) \textbf{S}tands \textbf{F}or \\
\textbf{LAH} & \textbf{L}ist \textbf{A}bbreviations \textbf{H}ere \\

}

%% ----------------------------------------------------------------
\clearpage  % Start a new page
\lhead{\emph{Constants}}  % Set the left side page header to "Physical Constants"
\listofconstants{lrcl}  % Include a list of Physical Constants (a four column table)
{
% Constant Name & Symbol & = & Constant Value (with units) \\
Speed of Light & $c$ & $=$ & $2.997\ 924\ 58\times10^{8}\ \mbox{ms}^{-\mbox{s}}$ (exact)\\

}

%% ----------------------------------------------------------------
\clearpage  %Start a new page
\lhead{\emph{Symbols}}  % Set the left side page header to "Symbols"
\listofnomenclature{lll}  % Include a list of Symbols (a three column table)
{
% symbol & name & unit \\
$a$ & distance & m \\
$P$ & power & W (Js$^{-1}$) \\
& & \\ % Gap to separate the Roman symbols from the Greek
$\omega$ & angular frequency & rads$^{-1}$ \\
}
%% ----------------------------------------------------------------
% End of the pre-able, contents and lists of things


%% ----------------------------------------------------------------
\mainmatter	  % Begin normal, numeric (1,2,3...) page numbering
\pagestyle{fancy}  % Return the page headers back to the "fancy" style

% Include the chapters of the thesis, as separate files
% Just uncomment the lines as you write the chapters

\chapter{Literature Review}

Single cell sequencing has increasing popularity in recent years. It is a technology that provides gene counts for each cell present in the sequencing. As the special count data ,  
The analysis of single cell RNA seq has its own workflows and challenges. A typical workflow starts from pre-processing steps of the raw count matrix. Poor quality cells and genes are filtered in the quality control steps. 
Single-cell data has technical noise and is therefore noisy. 

Genes and cells that pass the quality control can be used for downstream analysis. 
One common cell-level analysis is to find cell types in the data. Cells are often clustered based on similarities and different methods are used to annotate the clusters. 
The computed clusters are useful for further cell type annotations.

There are two main approaches to annotate the computed clusters. 
approach is to identify marker genes that are representative for each cluster, and further analysis with marker gene sets will reveal more insights about the cell identity. 
Although the automatic cluster annotation methods exist, manual annotation remains its significance when the experiment is performed under different conditions and the reference cells are not applicable. 
As the number of marker genes tend to be large, the obtained marker genes are typically examined as a group. The group of marker genes is tested against many well-known gene sets to identify over-represented biological pathways or processes involved in. The biological pathways will aid the understanding of existing cell identities.  

Marker analysis 
Manual cell type annotation involves finding marker genes that are highly regulated for the cluster of interest. Marker genes are identified by performing differential expression testing between the cells in the interested cluster versus all the rest cells. A considerable amount of literature has been published on marker-gene identification. There are various statistical tests used in DE, ranging from basic statics such as t-statistic to . 
The literature comparing thirty-six DE methods has highlighted that current tools have significant variations in terms of the number of detected marker genes and their significances.  Tools developed originally for bulk RNA sequence, such as Limma, edgeR are argued to be competitive to those designed specific for single cell RNA sequencing. 
Gene set testing 
Different methods exist in the literature regarding gene set testing. 

Permutation-based methods  
As argued in , commonly used DE tools tend to give under-estimated p-vlaue estimates, resulting in an overestimation in the number of up-regulated genes.  
Single cell RNA seq usually contains large amount of cells in different cell identities, and permutation-based methods have the potential to approximate empirical distribution without specific distribution assumptions. P-value for each gene is expected to be more accurate with permutation testing, and therefore control the false positive rate in marker genes. 

Similarly for gene set testing, what many available tools tend to show is the underestimated p values for the genes. 
There is a relatively small body of literature that is concerned with applying nonparametric methods to find marker genes [] []. However, the time and design complexity for those methods possibly limits the practical applications. 

The key purpose of this project is to develop a permutation-based framework that performs marker identification to discover underlying cell types in the data. 
To make permutation-based framework work efficiently and effectively, different test statistics will be explored for marker gene analysis and gene set testing. It is aimed to provide more accurate p-value estimates with the consideration of cell-to-cell and sample-to-sample variabilities in a computationally efficient framework. The final permutation framework will be implemented in the speckle R package. 

\section{Standard workflow}

\section{Literature Review}


%\subsection{Dataset }



 % Introduction
Single cell sequencing has increasing popularity in recent years. It is a technology that provides gene counts for each cell present in the sequencing. As the special count data ,  
The analysis of single cell RNA seq has its own workflows and challenges. A typical workflow starts from pre-processing steps of the raw count matrix. Poor quality cells and genes are filtered in the quality control steps. 
Single-cell data has technical noise and is therefore noisy. 

Genes and cells that pass the quality control can be used for downstream analysis. 
One common cell-level analysis is to find cell types in the data. Cells are often clustered based on similarities and different methods are used to annotate the clusters. 
The computed clusters are useful for further cell type annotations.

There are two main approaches to annotate the computed clusters. 
approach is to identify marker genes that are representative for each cluster, and further analysis with marker gene sets will reveal more insights about the cell identity. 
Although the automatic cluster annotation methods exist, manual annotation remains its significance when the experiment is performed under different conditions and the reference cells are not applicable. 
As the number of marker genes tend to be large, the obtained marker genes are typically examined as a group. The group of marker genes is tested against many well-known gene sets to identify over-represented biological pathways or processes involved in. The biological pathways will aid the understanding of existing cell identities.  
 
Marker analysis 
Manual cell type annotation involves finding marker genes that are highly regulated for the cluster of interest. Marker genes are identified by performing differential expression testing between the cells in the interested cluster versus all the rest cells. A considerable amount of literature has been published on marker-gene identification. There are various statistical tests used in DE, ranging from basic statics such as t-statistic to . 
The literature comparing thirty-six DE methods has highlighted that current tools have significant variations in terms of the number of detected marker genes and their significances.  Tools developed originally for bulk RNA sequence, such as Limma, edgeR are argued to be competitive to those designed specific for single cell RNA sequencing. 
Gene set testing 
Different methods exist in the literature regarding gene set testing. 

Permutation-based methods  
As argued in , commonly used DE tools tend to give under-estimated p-vlaue estimates, resulting in an overestimation in the number of up-regulated genes.  
Single cell RNA seq usually contains large amount of cells in different cell identities, and permutation-based methods have the potential to approximate empirical distribution without specific distribution assumptions. P-value for each gene is expected to be more accurate with permutation testing, and therefore control the false positive rate in marker genes. 

Similarly for gene set testing, what many available tools tend to show is the underestimated p values for the genes. 
There is a relatively small body of literature that is concerned with applying nonparametric methods to find marker genes [] []. However, the time and design complexity for those methods possibly limits the practical applications. 

The key purpose of this project is to develop a permutation-based framework that performs marker identification to discover underlying cell types in the data. 
To make permutation-based framework work efficiently and effectively, different test statistics will be explored for marker gene analysis and gene set testing. It is aimed to provide more accurate p-value estimates with the consideration of cell-to-cell and sample-to-sample variabilities in a computationally efficient framework. The final permutation framework will be implemented in the speckle R package. 

%\chapter{Assumptions}
\section{Assumption}
The experiments are conducted with simulated count data from Splatter which makes assumptions about the expected behaviour of true DEs. 
Splatter fits the real count matrix with a lognormal distribution and allows . The parameter de.facloc ad de.facscale control the mean and standard deviation of the assigned multipliers to the ramdomly selected true DEs. 
Although two-sided tests are popular in practice, one sided test will be used to identify up-regulated genes. The purpose of marker genes analysis is to identify up-regulated genes that can be represntative for specific cell types so that more clues can be made about existing cell types. Two-sided statistical test will include both up-regulated and down-regulated genes, and those down-regualted genes will cause ambigurity for further interpretation. 
\section{Limma Framework}
Limma fits a linear model for each gene. The adjusted p-values are believed to be over-optimistic. 
\section{Permutation Framework}
For any statistic $\textbf{T}_{obs}$ calculated from original count matrix, the same statistic $\textbf{T}_{perm}$ will be constructed from each permutated count matrix. The one-sided p-value will be evaluated based on the probability of having a larger permutated statitic than the observed statistic.\\
Let $N$ denote the number of permutations, the raw p-value $p$ for each gene will be 
\begin{align*} 
	p &=  \frac{\textbf{I}_{\textbf{T}_{perm} > \textbf{T}_{obs}} + 1}{N+1} 
\end{align*}
The key advantage of permutation test is the distribution-free property, which makes a wide range of statistics be and comparable.   
As the efficiency of limma framework, the permutation framework is implementated on top of the limma package, which makes the comparision simpler. 
%\subsection{Dataset }
\section{statistic}
t-statistic ($t$) and moderated t-statistic \modt are widely used for identifying marker genes in scRNA-seq. 
It is agued that \modt should be prefera
treat.t 
One additional statistic, $\textbf{W}, ~ \textbf{W} = \log \mathrm{FC} \times (1 - p) $ where $p$ is the raw p-value of the t-statistic. 
An ideal marker gene is expected to have a larger logFC, and a smaller p-value, which will generate a larger $W$ statistic.  % Background Theory 

%\chapter{Comparision}
\section{Splatter simulation}
Splatter is a competitve scRNA-seq simulator. 
There are diverse parameter sets that control different properties of the simulation, and the parameters of interest is related to DEs. 
Compare the simulation with the real data. 
To compare the simulated data with real data, two cell types are extracted from one individual separately. Splatter is used to learn a set of parameters for each cell type. 
Limma is applied to estimate the DE proportions for the selected cell types, and the calculated relative proportions of up and down regulated genes is manually set for simulation . 
The plot $3.1$ summaries the comparison between simulated data and the original data. It can be seen that the library size is estimated well . The cell type NK has a relatively low cell proportion in the original dataset, splatter may have thr risk . 
\section{Effect of Strength of DEs }
This section will focus on the effect of 
A closer . To make the simluation simple and ideal, additional changes are made to the simulated true DEs. It's expected that the true DE set for each cell type is unique, and the shared DEs acoess different groups are removed. Besides, it is noticed that some simulated true DEs are associated with negative logFC, in particular with small de.facloc and large standard deviation. Such true DEs are not realistic in practice and are filtered. 
The mean and variance of the multipliers to the DE genes have a joint effect on the difficulty level of marker identification. 
To keep the simulation simple for exploration, one largest cell type of one individual is extrated anf learned by Splatter. The experiment simulates two groups, each with 2\% DE genes. 5,000 genes and 1,000 cells
\section{Effect of different statistics}
%\subsection{Dataset }
Five different statistics are . 
t-statistic and moderated t-statistic are widely used for identifying marker genes in scRNA-seq. 
One additional statistic, W, is tested . 
The p-value of the t-statistic will be extremely small, has profound impact of the $W$ statistic 
\subsection{compare t and moderated t statistic}
\subsection{Parametric t statistic}
The null hypothesis of a t-test is that the logFC of a gene is equal to zero. There is a parametric version of a t-test which adds a threshold to the ordinary t-test. The parametric t-test has practical significance if  are particularly of interest. 
\subsection{ statistic}
Permutation tests do not have any assumption on the distrbutation of the test statistic, which makes it possible to extend to distinct statistics. 
This novel statistic, however, does not addtional improvements compared to the ordinary one-sided t-statistic. The reasons could be explained by the tolerance and of permutation test. The defined statistic shows a monotone . It is expected that true DEs will have large logFC and small p-values, and the product of logFC and $(1-p-value)$ will show a monotome behaviour as the ordinary t-statistic. The ranking of detected DEs by the W statistic will be generally the same as the ordinary t-statistic. \\
The W statistic serves as an example to illustrate the power of permutation tests. 
\section{Effect of permutation times}
The number of permutations $N$ sets a theoretical limit to the minimum possible raw p-value that can be achieved. When the permutation times is highly conserved, multiple testing adjustment may have the risk to . \\
The experiment is taken 
When all the raw p values are . 
\section{Effect of cell number proportions}
\subsection{}
 % Experimental Setup

%\input{Chapters/Chapter4} % Experiment 1

%\input{Chapters/Chapter5} % Experiment 2

%\input{Chapters/Chapter6} % Results and Discussion

%\input{Chapters/Chapter7} % Conclusion

%% ----------------------------------------------------------------
% Now begin the Appendices, including them as separate files

\addtocontents{toc}{\vspace{2em}} % Add a gap in the Contents, for aesthetics

\appendix % Cue to tell LaTeX that the following 'chapters' are Appendices

\chapter{An Appendix}

Lorem ipsum dolor sit amet, consectetur adipiscing elit. Vivamus at pulvinar nisi. Phasellus hendrerit, diam placerat interdum iaculis, mauris justo cursus risus, in 	% Appendix Title

%\input{Appendices/AppendixB} % Appendix Title

%\input{Appendices/AppendixC} % Appendix Title

\addtocontents{toc}{\vspace{2em}}  % Add a gap in the Contents, for aesthetics
\backmatter

%% ----------------------------------------------------------------
\label{Bibliography}
\lhead{\emph{Bibliography}}  % Change the left side page header to "Bibliography"
\bibliographystyle{unsrtnat}  % Use the "unsrtnat" BibTeX style for formatting the Bibliography
\bibliography{Bibliography}  % The references (bibliography) information are stored in the file named "Bibliography.bib"

\end{document}  % The End
%% ----------------------------------------------------------------